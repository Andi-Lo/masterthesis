\thispagestyle{empty}

  \begin{abstract}
  In einer Zeit, in der Städte aufgrund von zunehmendem Individualverkehr und damit einhergehender Umweltverschmutzung mehr und mehr an ihre Grenzen stoßen, wird die Verbesserung der Öffentlichen Verkehrsinfrastruktur immer wichtiger. Die Visualisierung von Daten des Öffentlichen Nahverkehrs kann hierbei einen Beitrag leisten, indem sie bspw. Pendlern und Reisenden als anschauliche Informationsquelle oder Städteplanern und Verkehrsunternehmen als Analyseinstrument und Planungsgrundlage dient. 
  Im Rahmen dieser Arbeit wurde ein Ansatz zur Visualisierung von Nahverkehrsdaten im GTFS-Format auf einer interaktiven Karte entwickelt. Das Ergebnis ist eine Live-Karte für den Raum Stuttgart, mit der auf einfache und praktikable Weise die Daten aller Fahrpläne des Verkehrsverbunds einsehbar werden. 
  \end{abstract}

  \pagebreak