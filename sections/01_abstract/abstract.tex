\thispagestyle{empty}

  \selectlanguage{ngerman}
  \begin{abstract}
    Die Städte stoßen immer mehr an die Grenzen von Verkehr und Umweltverschmutzung. Während die Welt nach Lösungen sucht, die die Mobilitätsfrage beantworten, investieren Städte und Regierungen viel Geld in die Instandhaltung und Verbesserung der öffentlichen Verkehrsinfrastruktur. Zwar gibt es in dieser Hinsicht durchaus Verbesserungen, aber die Aufgabe, Pendlern und Reisenden die notwendigen Informationen zu geben, fehlt noch immer. Tausende von Menschen verlassen sich täglich auf die öffentlichen Verkehrssysteme. Dabei ist die Suche nach den richtigen Informationen in einem immer komplexer werdenden öffentlichen Verkehrsnetz eine Herausforderung. Wir sind immer noch auf gedruckte Fahrpläne und Kartenmaterialien angewiesen, die in der Nähe eines Bahnhofs gefunden werden können. Das Problem bei gedruckten Karten liegt darin, dass sie keine Echtzeit-Informationen enthalten und oft nicht so leicht zu lesen sind. Die Entwicklung und Visualisierung effektiver Methoden zur Erforschung dieser Systeme ist keine leichte Aufgabe, aber dennoch unverzichtbar, wenn es darum geht, Fortschritte zu erzielen.

    Diese Arbeit zeigt einen Ansatz zur Visualisierung von Verkehrsdaten auf einer interaktiven Karte mittels GTFS-Datenformat. Dabei wird eine Live-Karte für den Raum Stuttgart vorgestellt, mit der auf einfache und praktikable Weise die Daten eines Fahrplans einsehbar werden. Es wird analysiert, auf welche weise diese Daten in Kombination mit einer Karte genutzt werden können, um das Benutzererlebnis zu verbessern und neue Visualisierungsmöglichkeiten zu finden.
  \end{abstract}

  \selectlanguage{english}
  \begin{abstract}
    Cities are more and more reaching a limit in regards of traffic and pollution. While the world is searching for solutions that are answering the mobility question, cities and governments are investing lots of money for maintenance and betterments in the public transport infrastructure. While there are definitely improvements in those regard, the task of giving commuters and travelers the informations they need is still lacking behind. Thousands of people rely on Public Transport Systems every day but still struggle to find proper informations in an ever increasing complexity of public transport networks. We still rely on printed maps and card materials like tube maps that you can find near a station. The problem with printed maps is that they don't provide temporal information and they often are not so easy to navigate. Designing and visualizing effective methods to explore those systems is not an easy task, yet still essential in moving forward.

    This work will shows an approach of visualizing public transit data on an interactive map using the GTFS data format. We analyze how to use that data in combination with a map to increase the user experience and try to find new ways of visualizing them. We want to create a live map that lets you explore schedule data in a feasable and easy way.
  \end{abstract}
  \selectlanguage{ngerman}

  \pagebreak