\thispagestyle{empty}

% \begin{abstract}
  \section*{Abstract}
    Cities are more and more reaching a limit in regards of traffic and pollution. While the world is searching for solutions that are answering the mobility question, cities and governments are investing lots of money for maintenance and betterments in the public transport infrastructure. While there are definitely improvements in those regard, the task of giving commuters and travelers the informations they need is still lacking behind. Thousands of people rely on Public Transport Systems (PTS) every day but still struggle to find proper informations in an ever increasing complexity of public transport networks. We still rely on printed maps and card materials like tube maps that you can find near a station. The problem with printed maps is that they don't provide temporal information and they often are not so easy to navigate. Designing and visualizing effective methods to explore those systems is not an easy task, yet still essential in moving forward.

    This work will shows an approach of vizualising public transit data on an interactive map using the GTFS data format. We analyze how to use that data in combination with a map to increase the user experience and try to find new ways of visualizing them. We wan't to create a live map that lets you explore schedule data in a feasable and easy way.

  \pagebreak

% \end{abstract}