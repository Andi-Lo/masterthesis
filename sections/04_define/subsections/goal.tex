\subsection{Zielsetzung}
\label{sub:zielsetzung}
  In dieser Arbeit wird macOS Sierra 10.12.6 auf einem Macbook Pro 2,9 GHz Intel Core i5 mit 16 GB 1867 MHz DDR3 und Intel Iris Graphics 6100 1536 MB eingesetzt. Alle Messungen wurden auf dieser Maschine ausgeführt.\\

  Ziele ergeben sich sowohl hinsichtlich der Performance der einzelnen Komponenten (Datenbank, Server, Client) als auch visuell durch das erkunden von verschiedenen Lösungsansätzen. Es soll eine interaktive Karte entwickelt werden, die den öffentlichen Nahverkehr des Verkehrsverbunds Stuttgart-VVS auf einer Live Karte aufzeigt. Dafür sollen verschiedene Visualisierungsansätze in einem Prototypen entwickelt werden um am Schluss einen ganzen Ideen- bzw Visualisierungskatalog herauszuarbeiten.\\

  Für das Backend wird die Datenbank PostgreSQL und Nodejs verwendet. Nodejs ist nicht nur einfach aufzusetzen, sondern auch sehr performant und effizient für Web Applikationen einsetzbar. Zudem lässt es sich sehr einfach mittels Docker in der AWS (Amazon Web Services) Cloud veröffentlichen. Die Datenbank soll ein GTFS Feed der Stuttgart-VVS aufnehmen und dessen Daten in $0$ bis $200ms$ bereitstellen können. Des weiteren soll der Nodejs Server die Daten innerhalb von maximal $100ms$ verarbeitet haben. Das Frontend soll die Vehicle mit 60 FPS rendern können.

% subsection zielsetzung (end)