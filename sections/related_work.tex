\begin{newpage}
  \section{Related Work}
  \label{sec:related_work}
    Es existieren mehrere Publikationen, welche für diese Arbeit Relevanz haben oder die für dieses Thema wertvolle Informationen bereitgestellt haben. Die wichtigsten werden in diesem Abschnitt vorgestellt.\\

    Einen ersten guten Überblick zum momentanen Stand von Transportvisualisierungen bietet das Paper \textit{"`Visualizing Public Transport Systems: State-of-the-Art and Future Challenges"'}\parencite{marchi} von Massimo De Marchi. In seiner Arbeit wird dargelegt, welche visuellen Lösung bereits bestehen und welche stärken / schwächen sie haben. Dabei wird auf Zwei unterschiedliche Benutzertypen eingegangen. Namentlich als "`Traveler"' und "`Transportation Researcher"' genannt. Er beschreibt dabei welche verschiedenen Eigenschaften die jeweilige Nutzerrolle besitzt und welche Bedürfnisse diese haben. Diese Frage \textit{"`Wer soll die interaktive Karte am ende benutzen?"'} ist auch bei der Erstellung dieser Arbeit von entscheidender Bedeutung gewesen. Je nach Typ bedarf es einem anderen Schwerpunkt gerecht zu werden. In dieser Arbeit sollen vor allem diejenigen einen Nutzen daraus ziehen können, die\\

    \parencite{mbtaviz} beschreibt sehr gut die Vorgehensweise und verwendeten Technologien für die Umsetzung der Visualisierung von \textit{"`An interactive exploration of Boston's subway system"'}\footnote{\url{http://mbtaviz.github.io/}}. Es werden dabei sowohl die verwendeten Tools beschrieben als auch Mockups gezeigt die eindrücke in den Arbeitsprozess gewähren. Die resultierenden Visualisierungen und deren Detailgrad sind sehr Eindrucksvoll und lassen einen tiefen Einblick in das Verkehrsnetz zu.\\

    Eine weitere wertvolle Quelle ist die Arbeit von Patrick Brosi \textit{"`Real-Time Movement Visualization of Public Transit Data"'}\parencite{brosi}. Brosi stellt dabei die Entwicklung von Travic vor, der bereits eingangs\footnote{siehe Kapitel \ref{travic}} erwähnt wurde. Seine Arbeit zeigt einen Lösungsweg für die Visualisierung von Tausenden Fahrzeugen auf einer interaktiven Karte. Interessant sind dabei die diskutierten Vor- und Nachteile von verschiedenen Lösungsansätzen.
    Während seine Arbeit keine Datenbank verwendet, sondern alle Daten in den RAM-Speicher des Servers läd, um schnelle Zugriffszeiten zu erreichen, soll in unserer Arbeit eine Postgresql Datenbank zum Einsatz kommen. 
    Auch hat unsere Arbeit nicht den Anspruch eine weltweite Live Karte zur Verfügung zu stellen, sondern fokussiert sich lediglich auf den Raum Stuttgart unter Verwendung eines GTFS Feeds des Verkehrsverbundes Stuttgart-VVS. 
    Der Kern der Arbeit beschäftigt sich tiefer mit dem finden von neuen Visualisierungsansätzen und UI-Komponenten. Es steht also nicht die Entwicklung eines fertigen Produkts im Vordergrund, sondern viel mehr darum, wie Nutzern auf visuellem Weg Informationen auf interessante Weise präsentieren werden können.

  % sec:related_work (end)

\end{newpage}