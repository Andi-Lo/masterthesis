\subsubsection{GTFS Optimierungen}
\label{ssub:gtfs_optimierungen}

  Der erste Schritt um die Performance zu verbessern, ist die Optimierung von GTFS Feeds. Damit lässt sich die Datenmenge bereits vor dem Importieren in die Datenbank, erheblich verringern. Ein Tool um ein GTFS Feed umfassend zu optimieren ist \texttt{gtfstidy} \url{https://github.com/patrickbr/gtfstidy}. Es bietet dabei allerdings nicht nur die Möglichkeit für die Vereinfachung von Polylines sondern kommt mit einer ganzen Reihe an Optimierungsmöglichkeiten. 

  Der Kommandozeilenbefehl \colorbox{materialGrey}{\texttt{\color{white}{\$ gtfstidy -sSiRDeO input.zip output}}} optimiert das Stuttgart-VVS Feed wie folgt:
  
  \begin{itemize}[label={}]
    \item \textbf{-s} Reduziert die Punktanzahl einer Polyline
      
    \item \textbf{-S} Entfernt redundante Polylines.

    \item \textbf{-i} Umwandlung von Zeichen ID's (String) in Zahlen ID's (Integer).\footnote{Aus der String ID \texttt{'1.T0.10-1-j17-1.16.H'} wird \texttt{78}}

    \item \textbf{-O} Entfernt Feed Einträge die nicht referenziert werden.

    \item \textbf{-R} Entfernt doppelt vorhandene Routen.

    \item \textbf{-e} Setzt fehlerhafte oder optionale Felder auf einen Standard Wert.

    \item \textbf{-D} Entfernt fehlerhafte Einträge aus dem Feed.
  \end{itemize}

  Durch Verwendung von gtfstidy konnte das Feed optimiert werden und die Datengröße der einzelnen Dateien um folgendes Maß verringert werden.

  \begin{longtable}{|>{\raggedright \arraybackslash}p{5.0cm}|>{\raggedright \arraybackslash}p{5.0cm}|>{\raggedright \arraybackslash}p{5.0cm}|}
    \hline
    Dateiname & Größe davor& Größe danach\\
    \hline
    trips.txt & 6 MB & 2.8 MB\\
    stop\_times.txt & 103 MB & 53 MB\\
    stops.txt & 651 KB & 355 KB\\
    shapes.txt & 77.3 MB & 22.4 MB\\
    routes.txt & 54 KB & 38 KB\\
    calendar\_dates.txt & 557 KB & 463 KB\\
    \hline
    \caption{Tabellengröße bevor und nach anwenden von gtfstidy}
    \label{tbl:gtfs_tidy_results}
  \end{longtable}

  Insgesamt konnte so die Größe des Feeds von 79 MB auf 118 MB um knapp 50\% verringert werden. Vor allem die Umwandlung von langen String ID's in kürzere Integer ID's trägt maßgeblich zur Verringerung der Dateigröße bei.

% subsubsection gtfs_optimierungen (end)