% GTFS format not really helpful for this kind of application (cause no need of writing data butreading).

\begin{newpage}
  \section{Fazit}
  \label{sec:fazit}
    In dieser Arbeit konnte gezeigt werden, wie durch die Verwendung eines GTFS Feeds eine Live Visualisierung für die Region Stuttgart erstellt werden konnte. Dabei konnten die im Zielsetzungsteil gesteckten Ziele erreicht werden. Den Mehrwert einer solchen Karte ergibt sich vor allem im verkehrsanalytischen Bereich. Der Einsatz einer Live Karte wurde bereits in vielen Bereichen adaptiert und für viele Bereiche stellt er einen echten Mehrwert für den Nutzer dar. Die Herausforderung bei der Abbildung des öffentlichen Nahverkehrs wurde allerdings auch in dieser Arbeit noch nicht gemeistert. So stellt vor allem die schiere Anzahl an Vehicles auf der Karte ein Problem dar. Dabei geht die Übersicht verloren und der Anwender kann nicht mehr genau erkennen, wo sich sein Zug oder Bus denn befinden soll. Eventuell müsste eine Live Karte spezifischer auf die einzelnen Bereiche zugeschnitten sein. Eine Live Karte die einen U-Bahnplan darstellt, hätte bei weitem weniger erzeugtes "`Rauschen"' durch andere Vehicle und so kann der Fokus des Anwenders direkt auf die für ihn relevanten Bereiche gelenkt werden. Für eine Web-Applikation könnte ein art Live Monitur für eine bestimmte Station oder Linie ein guter Ansatz sein.\\

    Als Ausblick kann gesagt werden, dass für dieses Projekt mehrere Verbesserungen möglich sind. Die Verknüpfung mit einer Echtzeitkomponente bietet am meisten Potential für weiterführende Gestaltungen. Die verschiedenen Stadien, die ein Vehicle annehmen kann (verspätet, verfrüht, pünktlich) könnten in die verschiedenen UI-Elemente mit einfließen als auch auf der Karte in kreativer Weise verarbeitet werden. Beispielsweise könnte ein Vehicle eine gewisse Farbe annehmen, wenn dies Verspätung hat. Aber auch statistische Auswertungen könnten dem Anwender angezeigt werden. Möglich wäre dabei das Anzeigen der Verspätung im vergleich zum Vortag oder über einen gewissen Zeitraum als Diagramm. Auch könnte eine Mögliche "`Soll"'-Position angezeigt werden, wo das Vehicle laut Fahrplan eigentlich sich befinden müsste.

    Auch der "`Wegfinder"' aus Kapitel \ref{ssub:wegfindung} hat Potential für umfassende Erweiterungen. Nicht nur die Erweiterung des Routing Algorithmus durch die Hinzunahme von verschiedenen anderen Daten wie zum Beispiel Verkehr oder Echtzeitdaten, sondern auch die Möglichkeit einer visuellen Wegfindung bietet viel Raum für kreative neue Ideen. Dabei müsste wohl Tests mit echten Anwendern zeigen, in wie weit diese Möglichkeit anklang findet oder in Wahrheit vielleicht doch eher ablehnung finden würde.

  % section fazit (end)
\end{newpage}