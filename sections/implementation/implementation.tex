\begin{newpage}
\section{Implementierung}
\label{sec:implementierung}
  In dieser Arbeit wird ein Macbook Pro 2,9 GHz Intel Core i5 mit 16 GB 1867 MHz DDR3 und Intel Iris Graphics 6100 1536 MB eingesetzt.
  Für das Backend wird die Datenbank PostgreSQL und Nodejs verwendet. Nodejs ist nicht nur einfach aufzusetzen, sondern auch sehr performant und effizient für Web Applikationen einsetzbar. Zudem lässt es sich sehr einfach mittels Docker in der AWS (Amazon Web Services) Cloud veröffentlichen.\\

  Dieses Kapitel ist in zwei Abschnitte unterteilt. Namentlich \texttt{Backend} und \texttt{Frontend} genannt. Im Abschnitt Backend sollen die Implementierungsschritte für das Erstellen eines Backends beschrieben werden, dass ein GTFS-Feed verarbeiten und dessen Daten dem Frontend zur Verfügung stellt. Der Abschnitt Frontend beschreibt verschiedene Performance optimierungen und die einzelnen UI-Komponente mit deren Funktionensweise.

  \subsection{Backend}
\label{sub:backend}
  Da der GTFS Standard eine fertige relationale Beziehung der einzelnen Dateien festlegt, ist der Einsatz einer relationalen Datenbank sehr naheliegend. Dabei gibt es eine breite Palette an Auswahl. Damit die Anwendung möglichst Zugänglich bleibt, liegt der Fokus auf Datenbanken die unter einer Open-source Lizenz kostenfrei zur Verfügung stehen. Die zwei populärsten sind MySQL und PostgreSQL\parencite{db_engines}. Beide haben ihre Vor- und Nachteile und die Entscheidung ist mehr eine persönliche Präferenz, als ein großer Vorteil des Einen über den Anderen. Einen kleinen Vorteil bietet PostgreSQL's Unterstützung für Array-Types, welche sehr hilfreich beim speichern und abfragen von Daten ist. Ansonsten ist dieses Projekt auch mit MySQL realisierbar.\\

  In diesem Abschnitt soll beschrieben werden, welche Probleme beim Arbeiten mit dem GTFS Datenformat und einer PostgreSQL Datenbank bestehen und wie diese gelöst wurden. Dabei steht vor allem die Behebung von Engpässen bei der Query\footnotemark Performance, als auch die Verringerung der zu verarbeitenden Datenmenge.

  \footnotetext{Ein Query ist eine Informationsanfrage an eine Datenbank} 
  
  % Gliederung eventuell nochmals beschreiben.

  \subsubsection{GTFS - Probleme und Herausforderungen}
\label{ssub:gtfs_probleme_und_herausforderungen}
   Bereits 1993 stellte \texttt{Jakob Nielsen} einen Richtwert für die Antwortzeit einer schnellen Webanwendung vor:

  \begin{quote}
    \textit{"`1.0 second is about the limit for the user's flow of thought to stay uninterrupted, even though the user will notice the delay. Normally, no special feedback is necessary during delays of more than 0.1 but less than 1.0 second, but the user does lose the feeling of operating directly on the data."'}\parencite{nielsen}
  \end{quote}

  Ob oder wie dies Antwortzeit erreicht werden kann, soll hier nicht vertieft werden, denn zu diesem Thema habe ich bereits meine Bachelorarbeit gewidmet\parencite{lorer}.

  Damit eine Webanwendung aber überhaupt eine Chance hat, diese Geschwindigkeitsmarke zu erreichen, ist eine schnelle Antwortzeit des Backends sehr wichtig. Dabei sind Antwortzeiten Innerhalb von 0 bis 200 Millisekunden ein sehr guter Wert. Natürlich gilt: Je weniger umso besser. Diese Benchmark mit dem GTFS Format zu erreichen war eine der Hauptherausforderungen dieser Arbeit.\\

  Das GTFS Format hat den entscheidenden Nachteil, dass es eine hohe Komplexität aufweist, sobald Daten aus verschiedenen Tabellen benötigt werden. Für eine Live Visualisierung, sind Daten aus nahezu allen Tabellen relevant. Abbildung~\ref{fig:gtfs_joined_tables} zeigt, welche davon benötigt - beziehungsweise nicht benötigt werden (grau).

  \begin{figure}[ht]
    \begin{center}
      \includegraphics[width=\textwidth]{gtfs_joined_tables.jpg}
      \caption{Benötigte GTFS Tabellen\parencite{google_gtfs_reference}}
      \label{fig:gtfs_joined_tables}
    \end{center}
  \end{figure}

  Das UML Diagramm ist auf den ersten Blick relativ simpel zu verstehen und die Grundlagen der verschiedenen Relationen wurde bereits in Kapitel~\ref{ssec:gtfs_datenformat} beschrieben. Wo liegt also das Problem? In Worten ließe sich diese Datenbankabfrage mit folgendem Statement beschreiben: 

  \begin{quote}
    \label{query_statement}
    \textit{"`Gib uns alle aktiven Trips mit deren Linienverlauf, die am heutigen Tag aktiv sind und in einer Zeitspanne zwischen $t_a$ und $t_b$ liegen."'}
  \end{quote}

  Das große Problem dieses Satze liegt in der Zeitkomponente \textit{"`Trips die am heutigen Tag aktiv sind zwischen ..."'}. Die Trip Tabelle selbst (bezogen auf Abbildung~\ref{fig:gtfs_joined_tables}), hat dies bezüglich keinerlei Informationen darüber. Auch die Calendar- und Calendar-Dates Tabelle beinhaltet nur Informationen, an welchem Datum ein Trip stattfindet, nicht aber um welche Uhrzeit. 

  Erst die Stoptime Tabelle ermöglicht es uns, eine Aussage zu treffen, wann ein Trip aktiv ist. Über die zwei Felder \texttt{arrival\_time} und \texttt{departure\_time} lässt sich sagen, zu welchem Zeitpunkt ein Vehicle an einer Station anhält. Die erste und letzte Station ($S_1$ und $S_n$) geben uns also einen zeitlichen Rahmen, in dem der Trip aktiv ist.
  Hierbei wird klar, dass allein die Beantwortung der Frage zur zeitlichen Komponente, bereits sehr viele Daten aus verschiedenen Tabellen benötigt. Die anderen Tabellen wie \texttt{shape}, \texttt{route}, \texttt{stop} und \texttt{frequency} würden für weitere Informationen wie Vehicle Farbe, Stop Position (Längen- und Breitengrad) oder den Linienverlauf (shape) benötigt werden. Um an die Daten zu gelangen, müssen alle benötigten Tabellen mittels SQL \texttt{JOIN} miteinander verknüpft werden. Dies geschieht durch die Verbindung der einzelnen Reihen zweier Tabellen (TabelleA und TabelleB) gegen eine Verknüpfungsbedingung. Das Resultat ist eine neue Ergebnistabelle mit den Inhalten der kombinierten Reihen. Solche Verknüpfungen sind besonders dann Zeitintensiv, wenn eine große Menge an Daten (siehe Tabelle:~\ref{table:table_metrics}) kombiniert werden. Die Metriken der Tabellen sind dabei wie folgt:

  \begin{longtable}{|>{\raggedright \arraybackslash}p{5.0cm}|>{\raggedright \arraybackslash}p{5.0cm}|>{\raggedright \arraybackslash}p{5.0cm}|}
  \caption{Tabellen Metriken} \label{table:table_metrics}\\
    \hline
    Tabellen Name & Anzahl Reihen\\
    \hline
    trips.txt & 71,000\\
    stop\_times.txt & 1,3000,000\\
    stops.txt & 7,900\\
    shapes.txt & 1,085,860\\
    \hline
  \end{longtable}
  
  Die für das oben genannte Statement~\ref{query_statement} äquivalente SQL-Abfrage ist aufgrund seiner Länge (113 Zeilen Code) im Anhang unter Listing~\ref{lst:get_active_trips_query} zu finden. Diese SQL Abfrage ist allerdings nicht Performant. Sollen alle Trips in einem Zeitraum von 1 - 15 Minuten gefunden werden, sind bereits Rechenzeiten entstanden, die aufgrund ihrer langen Laufzeit abgebrochen werden mussten. In mehreren Iterationen wurde versucht die SQL-Abfrage zu optimieren, was allerdings keine Verbesserung herbeiführte. Es sind zu viele JOIN Verknüpfungen und WHERE Bedingungen in dieser Abfrage, als das sich eine Performante Lösung damit finden lässt. Es musste ein neuer Ansatz gefunden werden um Abfragezeiten erheblich zu verringern.
  
% subsubsection gtfs_probleme_und_herausforderungen (end)
  \subsubsection{GTFS optimieren}
\label{ssub:gtfs_optimieren}
  In diesem Abschnitt soll erklärt werden, welche Schritte nötig waren, um eine schnelle Abfrage von GTFS Daten in einer Postgresql Datenbank zu ermöglichen.\\

  Das GTFS Format hat den Nachteil, dass es eine sehr hohe komplexität aufweißt, sobald Daten aus verschiedenen Tabellen benötigt werden. Eine Datenbankabfrage wie: \textit{"`Gib uns alle Trips die in einer Zeitspanne zwischen $t_a$ und $t_b$ liegen"'} führt zu einer Situation, in der verschiedenste Tabellen mittels SQL \texttt{JOIN} miteinander verknüpft werden müssen. Solche Verknüpfungen sind besonders dann Zeitintensiv, wenn eine große Menge an Daten in den jeweiligen Tabellen verarbeitet werden muss.

  Bei einem \texttt{JOIN} werden die einzelnen Reihen zweier Tabellen (TabelleA und TabelleB) über die JOIN-Bedingung miteinander Verknüpft. Das resultat ist eine neue Ergebnistabelle mit den kombinierten Reihen aus beiden Tabellen. Vor allem ein \texttt{INNER-JOIN} kann erhebliche Performance-Probleme bereiten:

  \begin{quote}
    \textit{"`Conversely, an inner join can result in disastrously slow performance or even a server crash when used in a large volume query in combination with database functions in an SQL Where clause."'}\parencite{wikipedia_sql_join}
  \end{quote} 

  die Abfrage bis zu \~20 Sekunden benötigt.

  % Use explain analyze on old queries to better make a statement about performance issues: https://robots.thoughtbot.com/postgresql-performance-considerations
  
  Dies ist für eine Anwendung natürlich nicht tragbar und es mussten Wege gefunden werden, die Abfragzeiten erheblich zu verringern.\\

  Ein Schritt in diese Richtung ist die Optimierung der GTFS Daten, bevor diese in Abschnitt~\ref{sub:daten_importieren} in die Datenbank geladen werden.

  The GTFS format has the draw back of increased complexity as soon as you need data from different tables.

  To ask the database for "provide all trips that lay inside of a given timeframe" yielded a situation where query times reached above ~20 seconds.

  For the query to execute you need several expensive joins of database tables. This forced the action of improving database performance by applying different optimizations. The graphics shows what kind of tables I would need for a complete query that gives me everything that I would need for doing a vehicle simulation on a map:

  \begin{itemize}
    \item Warum gtfs optimierung sinn macht
    \item Stuttgart-VVS feed IDs 
    \item GTFStidy verwenden -> ergebnisse aufzeigen?
  \end{itemize}
% subsubsection gtfs_optimieren (end)
  \subsubsection{Polyline optimieren}
\label{ssub:polyline_optimieren}

  Die Optimierung der Polyline ist ein sehr wichtiger Aspekt in meiner Arbeit und soll in diesem Abschnitt vertieft werden.

  \subsubsection*{Ramer–Douglas–Peucker}
  \label{ssub:ramer_douglas_peucker}
    Das Problem: Die im Stuttgart-VVS Feed zur Verfügung gestellten Polylines sind Überdefiniert und können aus tausenden Punkten bestehen. Für eine Visualisierung ist eine solche Genauigkeit nicht notwendig und aufgrund der großen Datenmenge problematisch. Im vorigen Abschnitt \ref{ssub:gtfs_optimierungen} wurde die Option \colorbox{materialGrey}{\texttt{\color{white}{-s}}} vorgestellt. Dieser Befehl verwendet den "`Ramer–Douglas–Peucker"' (RDP) Algorithmus um die Anzahl der Punkte einer Polyline zu reduzieren. Der Vorteil besteht darin, dass dabei nicht der Linienverlauf verändert wird. Abbildung~\ref{fig:simplify} zeigt ein Beispiel einer solchen Vereinfachung mittels einer JavaScript Bibliothek\footnote{Simplify.js \url{http://mourner.github.io/simplify-js/}}.

    \begin{figure}[htbp]
      \centering
      \subfloat[Polyline vor RDP]{\includegraphics[width=0.48\textwidth]{simplify_before.jpg}\label{fig:simplify_before}}
      \hfill
      \subfloat[Polyline nach RDP]{\includegraphics[width=0.48\textwidth]{simplify_after.jpg}\label{fig:simplify_after}}
      \caption{Vereinfachung einer Polyline mittels Simplify.js}
      \label{fig:simplify}
    \end{figure}

    Ausgangspunkt ist eine Polyline mit $\approx1000$ Punkten (\ref{fig:simplify}a). Nach der Vereinfachung (\ref{fig:simplify}b) ist die Anzahl auf 100 Punkte reduziert, ohne dabei visuell merklich einzubüßen. Dies ist eine erhebliche Reduzierung der Punkte um 90\%. Wie wirkt sich dieser Algorithmus positiv auf das Projekt aus? Die Vorteile sind weitreichend. Sehen wir uns die Shape Tabelle in Abbildung \ref{fig:shape_simplify} an. \ref{fig:shape_simplify}a zeigt 394 Reihen vor der Optimierung und nur noch 140 (\ref{fig:shape_simplify}b) nach Anwendung von gtfstidy.

    \begin{figure}[htbp]
      \centering
      \subfloat[Shapte Tabelle vor RDP]{\includegraphics[width=0.48\textwidth]{shape_simplify_before.jpg}\label{fig:shape_simplify_before}}
      \hfill
      \subfloat[Shapte Tabelle nach RDP]{\includegraphics[width=0.48\textwidth]{shape_simplify_after.jpg}\label{fig:shape_simplify_after}}
      \caption{Reduzieren der Polyline via gtfstidy}
      \label{fig:shape_simplify}
    \end{figure}

    In seinem Originalzustand hat das verwendete VVS Feed 1,085,859 Mio Zeilen. Nach der Anwendung sind diese auf 617,653 Tsd. verringert. Testet man folgende PostgreSQL Abfrage
    \colorbox{materialGrey}{\texttt{\color{white}{{\color{materialBlue}SELECT} * {\color{materialBlue}FROM} gtfs\_shapes {\color{materialBlue}WHERE} shape\_id = {\color{materialRed}3201}}}}
    die alle Punkte einer Polyline ausgeben soll, so ergibt sich für ein optimiertes Feed eine Query Zeit von $\approx145 ms$ und für das nicht optimierte Feed $\approx250 ms$. Schon durch diese einfache Methode sind bereits erste Performance Steigerungen wahrnehmbar.

  % subsubsection ramer_douglas_peucker (end)

  \subsubsection*{Aggregieren der Shape Tabelle}
  \label{ssub:aggregieren_der_shape_tabelle}
    In GTFS wird für jeden Punkt einer Polyline eine Reihe in der Datenbank belegt. Diese Abfolge ist durch eine sogenannte \texttt{Shape Point Sequence} festgelegt, was nichts anderes ist als eine Zahl von $1$ bis $n$. Dies ist auch bereits in obiger Tabelle \ref{fig:shape_simplify} zu sehen gewesen. Sehr viel effektiver wäre es allerdings, diese Punkte nicht Reihenweise zu speichern, sondern alle zusammen gehörenden Punkte in einem einzigen Feld zu speichern. Dies ist in PostgreSQL durch eine Aggregierung möglich. Daraus ergibt sich folgende Shape Tabelle:

    \begin{figure}[htbp]
      \begin{center}
        \includegraphics[width=\textwidth]{aggregated.png}
        \caption{Aggregated Shape}
        \label{fig:aggregated}
      \end{center}
    \end{figure}

    Wie zu sehen ist benötigt nun eine Polyline in der Shape Tabelle nicht mehr 140 Reihen, sondern nur noch eine einzige. Für diese Arbeit ist dies auf alle Polylines angewendet worden und in einer neuen Tabelle namens \texttt{denormalized\_shapes} abgespeichert. Dadurch ist die Berechnung der Aggregierung nur einmal nötig. Der SQL-Befehl dafür ist dem Anhang unter \ref{lst:sql_aggregate_shape}. zu entnehmen.
    Wenden wir die selbe SQL Abfrage, die bereits oben Verwendung fand, auf die neue \texttt{denormalized\_shapes} Tabelle an. Die Query Zeit ist auf $\approx1ms$ gesunken! Anstatt hunderte Reihen muss nur eine einzige Reihe ausgelesen werden, was sehr sehr schnell ist. Durch das Denormalizing der Shape Tabell ist auch die Anzahl der Reihen auf ein Minimum gesunken. Von den früheren 617,653 Tsd. Reihen, sind jetzt durch die Aggregation nur noch 4,524 Tsd. übrig.
  
  % subsubsection aggregieren_der_shape_tabelle (end)

  \subsubsection*{Polyline Encoding}
  \label{ssub:polyline_encoding}
    Die letzte Maßnahme zur Optimierung der Polyline stellt das sogenannte Polyline Encoding dar. Auf das genaue Verfahren dieser Methode will ich hier an dieser Stelle allerdings nicht näher eingehen.

    Das Polyline Encoding kann in JavaScript beispielsweise durch das Google-Polyline\footnote{\url{https://www.npmjs.com/package/google-polyline}} Paket eingesetzt werden. Durch das Encoding wird eine Polyline, bestehend aus Punkte in Form eines Arrays, in eine Textform umgewandelt. So können zum Beispiel die Punkte: (38.5, -120.2), (40.7, -120.95), (43.252, -126.453) bzw als Array $\Rightarrow$  [[38.5, -120.2], [40.7, -120.95], [43.252, -126.453]] in
    \colorbox{materialGrey}{\texttt{\color{white}{\_p\textasciitilde iF\textasciitilde ps|U\_ulLnnqC\_mqNvxq`@}}}
    codiert werden. Dies geschieht in unserer Anwendung immer bevor Daten von Server in Richtung Client geschickt werden: Encode $\rightarrow$ Send $\rightarrow$ Decode. Da eine codierte Polyline weniger Zeichen benötigt, kann damit Datenvolumen bei der Komunikation zwischen Server und Client gespart werden.

  % subsubsection polyline_encoding (end)


% subsubsection polyline_optimieren (end)
  \subsubsection{Erstellen der Datenbank}
\label{ssub:erstellen_der_datenbank}
  init.sh
% subsubsection erstellen_der_datenbank (end)
  \subsubsection{Daten importieren}
\label{sub:daten_importieren}
  import.sh
% subsubsection daten_importieren (end)
  \subsubsection{Denormalisierung der Tabellen}
\label{ssub:denormalisierung_der_tabellen}
  \begin{itemize}
    \item What is denormalizing
    \item Why we needed to do it? (fucking slow queries: reference to old statement)
    \item What results did it yield
  \end{itemize}
% subsubsection denormalisierung_der_tabellen (end)
  \subsubsection{Konfigurierung}
\label{ssub:konfigurierung}
  \begin{itemize}
    \item Why optimize
    \item what can be optimize
    \item a-b test of configuration
    \item discuss test results
  \end{itemize}
% subsubsection konfigurierung (end)
% subsection backend (end)

  \begin{newpage}
  
  \subsection{Frontend}
  \label{sub:frontend} 
    Das Frontend besteht aus verschiedenen UI-Komponenten. Diese sollen in diesem Kapitel beschrieben und die wichtigsten Algorithmen zur Darstellung der Vehicle werden näher betrachtet.

    % TODO: rewrite section introduction

    \subsubsection{Client}
\label{ssub:client}
  Im Client findet die Visualisierung statt. Er stellt die erste Anfrage an den Server um alle Trips in einem Zeitraum zu bekommen. Die Antwort vom Server ist ein Objekt bestehend aus einer ID und einer \texttt{GeoJSON-FeatureCollection}. Die Verwendung von GeoJSON hat den Vorteil, dass verschiedene Bibliotheken für dieses Format zur Verfügung stehen, die dessen Verarbeitung vereinfacht. Auch Mapbox setzt auf die Verwendung von GeoJSON und ist fest damit verbunden. Der Nachteil von GeoJSON ist seine sehr wortreiche Beschreibung. Dass macht es zwar für Menschen gut lesbar, allerdings auf kosten der Datengröße.\\

  Für die Programmierung des Clients werden folgende Bibliotheken als die wichtigsten angesehen:

  \begin{itemize}[label={}]
    \item \textbf{Turf}\footnote{\url{http://turfjs.org/docs/}} Stellt eine ganze Reihe an Funktionen für die raumbezogene Verarbeitung von Daten zur Verfügung. Beispielsweise lassen sich mittels Turf unter anderem Distanzen, Flächen oder Schnittpunkte berechnen.

    \item \textbf{Mapbox-gl-js}\footnote{\url{https://www.mapbox.com/mapbox-gl-js/api/}} wird benötigt um das Kartenmaterial von Mapbox zu verwenden und bietet eine API dafür an.

    \item \textbf{Lodash}\footnote{\url{https://lodash.com/}} ist eine Hilfsbibliothek, die verschiedene Funktionen zur Verfügung stellt, die das Arbeiten mit JavaScript vereinfachen.

    % \item \textbf{Moment}\footnote{\url{http://momentjs.com/}} bietet das Validieren, Parsen, Manipulieren und Anzeigen von Zeitinformationen. Vor allem das Umwandeln von verschiedenen Zeitformaten erwies sich als hilfreich. 
    % maybe change to timeFormatter  https://www.npmjs.com/package/time-formater
  \end{itemize}

  % Todo: Eventuell doch auslagern in neue Datei + subsubsection?
  % Eventuell vehicle object und architektur ein wenig beschreiben? Vielleicht UML Diagramm? (mb zu aufwendig)


  \subsubsection*{Algorithmen}
  \label{ssub:algorithmen}
    Der wichtigste Algorithmen des Clients ist \texttt{AnimateVehicle}. Dieser solle in diesem Abschnitt in der Form von Pseudo-Code erläutert werden.\\

    Nachdem die angefragten Trips beim Client eingetroffen sind, wird ein Animation-Loop begonnen, der die Vehicle auf der Karte animiert. Dazu wird ein Algorithmus, der in Listing \ref{alg:animate_algorithmus} beschrieben wird, verwendet.

    \pagebreak
    \begin{algorithm}[H]
      \caption{Animate Vehicle}\label{alg:animate_algorithmus}
      \begin{algorithmic}[1]
        \Procedure{animateVehicle}{}
          \State ServerQueryTimer $\gets$ 30 Seconds
          \State Vehicles $\gets$ Vehicles Inside Bounding Box
          \State Trips $\gets$ Requested Trips
          \Function{animate}{timestamp}
            \ForAll{Vehicles as Vehicle} \State{
              \If{Vehicle started its Trip} 
                \State \Call{calculateVehiclePosition}{Vehicle}
              \EndIf
              \If{Vehicle not started its Trip}
                \State \Call{checkVehicleActivity}{Vehicle, Trips}
              \EndIf
              \State \Call{checkIfVehicleHasFinished}{Vehicle}
              \State \Call{updateMapWithNewPositions}{Vehicles}
            }\EndFor

            \If{ServerQueryTimer Expired} 
              \State Query Server for New Trips
              \State ServerQueryTimer $\gets$ 30 Seconds
            \EndIf

            \State \Call {animate}{timestamp}
          \EndFunction
          
        \EndProcedure
      \end{algorithmic}
    \end{algorithm}
   % subsubsection algorithmen (end)

   Innerhalb dieses Animation-Loops passieren mehrere Dinge. Zuerst wird geprüft ob sich ein Vehicle überhaupt im Sichtbereich des Anwenders befindet. Trifft das zu, wird für eben diese Vehicle die Distanzen berechnet und die Position des Vehicles entlang der Polyline interpoliert. Falls das Vehicle seinen Trip noch nicht begonnen hat, wird überprüft ob das immer noch der Fall ist. Anschließend werden alle Vehicle geprüft, ob sie ihren Trip erledigt haben. Danach werden die Karte mit den neuen Positionen der Vehicle aktualisiert. Während all dies geschieht, läuft ein Timer mit, der nach dem Ablaufen von 30 Sekunden den Server nach neuen Trips befragt.

% subsubsection client (end)
    \subsubsection{Verbesserung der Client Performance}
\label{ssub:verbesserung_der_client_performance}

  Damit die Animationen auf der Karte bei 60 FPS möglich sind, werden mehrere Optimierungsschritte ausgeführt.

  \begin{itemize}
    \item \textbf{Manipulieren der FPS:} Je niedriger das Zoom Level der Karte, umso geringer wird die FPS eingestellt. Das hat den Hintergrund, dass bei niedrigem Zoom, die Bewegung der Vehicle fast nicht mehr Wahrnehmbar ist. Wohingegen bei höherem Zoom die Animation umso flüssiger sein muss. Folgende Grenzen haben sich bei Tests als gute Werte erwiesen:
    \begin{itemize}
      \item Zoom Level < 12 $\rightarrow$ 1 FPS
      \item Zoom Level > 15 $\rightarrow$ 60 FPS
      \item Sonst $\rightarrow$ 8 FPS
    \end{itemize}
    Dieses Vorgehen bringt den Vorteil, dass bei niedrigem Zoom viel mehr Vehicle angezeigt werden, aber diese nur noch auf 1 FPS animiert werden müssen. Bei hohem Zoom ist dies genau umgekehrt. Dort sind nur noch wenige Vehicle sichtbar, diese werden aber dafür bei 60 FPS animiert. Somit ist einerseits eine gute Performance bei vielen Vehicles möglich und andererseits die Animation bei genauerer Betrachtung trotzdem sehr flüssig.

    \item \textbf{Speichern von Zuständen:} Um Rechenleistung einzusparen, werden wann immer möglich ausgerechnete Werte abgespeichert, damit diese nicht nochmals berechnet werden müssen. Zum Beispiel (siehe Abbildung \ref{fig:polyline_segments}) weiß das Vehicle, auf welchem Polyline Segment\footnotemark es sich befindet. 

     \footnotetext{Ein Segment sei in diesem Kontext ein gerader Linienteil der Polyline, bestehend aus zwei Punkten $A, B$.}

    \begin{figure}[htbp]
      \begin{center}
        \includegraphics[width=0.45\textwidth]{polyline_segments}
        \caption{Polyline Segmente}
        \label{fig:polyline_segments}
      \end{center}
    \end{figure}

    Das bedeutet, dass die Richtung des Vehicles nicht neu berechnet werden muss, solange es diesem Segment folgt. Erst wenn das Vehicle von Segment $A,B$ auf ein neues Segment $B,C$ übergeht, muss die Richtung neu berechnet werden.

    \item \textbf{Aufteilen der Vehicle in zwei Gruppen:} Da der User durch seinen Bildschirm meistens nur einen Teil der Vehicle zu sehen bekommt, sind die Vehicle in die Gruppen unterteilt. Namentlich seien sie als \texttt{Innerhalb} und \texttt{Außerhalb} benannt. Die Gruppe Innerhalb besitzt all diejenigen Vehicle, die sich im Sichtfeld des Anwenders befinden. Diese werden bei der Animation bevorzugt behandelt und erhalten die volle Rechenleistung. Die Gruppe Außerhalb liegt nicht im Sichtfelds und wird maximal jede Sekunde nach ihrer Aktivität geprüft. Dadurch bleibt auch diese Gruppe immer aktuell.    
   
  \end{itemize}

% subsubsection verbesserung_der_client_performance (end)
    \subsubsection{UI Components}
\label{ssub:ui_components}
  
% subsubsection ui_components (end)


  % subsection frontend (end)

\end{newpage}

% section implementierung (end)
\end{newpage}