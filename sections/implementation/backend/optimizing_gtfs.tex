\subsubsection{GTFS Optimierungen}
\label{ssub:gtfs_optimierungen}
  
  Ein Schritt um die Performance zu verbessern, stellt die Optimierung von GTFS Feeds dar. Zum Teil lässt sich die Datenmenge bereits vor dem Importieren in die Datenbank, erheblich verringern. Ein sehr gutes Tool um dies zu erreichen ist \texttt{gtfstidy} \url{https://github.com/patrickbr/gtfstidy}.\\

  Der Kommandozeilenbefehl \colorbox{materialGrey}{\texttt{\color{white}{\$ gtfstidy -sSiRDeO input.zip output}}} optimiert das Suttgart-VVS Feed wie folgt:

  \begin{itemize}
    \item -s Vereinfacht die Polyline % show douglas peucker simplification 
    
    \item -S shape duplicate remover

    \item Umwandlung von Zeichen ID's (String) in Zahlen ID's (Integer) % why does this matter?

    \item -O Feed is checked for entries that are not referenced anywhere. These entries are removed from the output.

    \item -R route duplicate remover

    \item -e If optional field values of feed entries have errors, this processors sets them to the default values specified in the GTFS standard.

    \item -D drop erroneous entries from feed

  \end{itemize}

% subsubsection gtfs_optimierungen (end)