\subsubsection{Backend Performance}
\label{ssub:backend_performance}
  In diesem Abschnitt soll das gesamte Backend (Server und Datenbank) evaluiert werden. Dazu werden sowohl die Zeit zum Abfragen der Datenbank, als auch die Zeit zur Datenverarbeitung auf dem Server betrachtet werden.

  \begin{longtable}{|>{\raggedright \arraybackslash}p{4.5cm}|>{\raggedright \arraybackslash}p{1.2cm}|>{\raggedright \arraybackslash}p{1.2cm}|>{\raggedright \arraybackslash}p{1.2cm}|>{\raggedright \arraybackslash}p{1.2cm}|>{\raggedright \arraybackslash}p{1.2cm}|>{\raggedright \arraybackslash}p{1.2cm}|}
  \caption{Backend Evaluation}\label{tbl:backend_evaluation}\\
    \hline
    Anz. Trips & 20 & 100 & 500 & 1000 & 5000 & 10000\\
    \hline
    Query Zeit (ms)        & 25 & 88 & 124 & 200 & 855 & 1631 \\
    Verarbeitungszeit (ms) & 2 & 27 & 40 & 142 & 226 & 435 \\
    Summe (ms)             & 27 & 115 & 164 & 342 & 1081 & 2066 \\
    \hline
  \end{longtable}

  In Tabelle \ref{tbl:backend_evaluation} sind die Datenwerte für verschiedene Abfragen aufgelistet. Die Werte ergeben sich aus dem Mittelwert der Laufzeit in 10 Durchläufen. Wie bereits in Kapitel "`\nameref{ssub:bewältigung_der_datenmenge}"' festgestellt:

  \begin{quote}
    \textit{"`In einer Minute [werden] minimal 0 und maximal 27 Vehicle aktiv. Im Schnitt starten 9 Vehicles pro Minute ihre Fahrt."'}\ref{ssub:bewältigung_der_datenmenge}
  \end{quote}

  Aus dieser Aussage plus den gemessenen Werten lässt sich folgende Schlussfolgerung ziehen: Bei einer Anzahl zwischen 20 und 100 Trips reagiert der Server innerhalb von 25 bis $120ms$. Da die meisten Trips in diese Spanne fallen, ist dieses Ergebnis am bedeutendsten. 

  Im Bereich von 100 bis 500 Trips ist eine Antwortzeit von 115 bis $164ms$ immer noch sehr gut. Diese Anzahl an Trips ist dann relevant, wenn die Applikation das erste mal aufgerufen wird und die Karte noch leer ist. In diesem Fall kann es sein, dass der Server (je nach Datum und Uhrzeit) zwischen 200 - 500 Trips verarbeiten muss. Aber selbst Abfragen von bis zu 10.000 Trips, was ungefähr einer Zeitspanne von einem Tag gleich kommt, sind immer noch innerhalb von 2 Sekunden verarbeitet.\\

  Abschließend kann gesagt werden, dass in dieser Arbeit ein performantes Backendsystem für eine Web Applikation entwickelt worden ist, welches Serveranfragen effizient be- und verarbeiten kann.

  % \ref{sub:bewältigung_der_datenmenge}

% subsubsection backend_performance (end)