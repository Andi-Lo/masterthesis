%!TEX root = ../main.tex

\begin{newpage}
	
	\section{Einleitung}
		
	\label{sec:Einleitung}
		Datenvisualisierung ist ein Thema, das nicht nur in jüngster Zeit sehr viel Zuwendung fand, sondern auch zur Analyse von Sachverhalten immer wichtiger wird. So lassen sich komplexe Zusammenhänge eines Systems, oftmals erst dann richtig begreifen, wenn wir alle möglichen Zustände davon erfassen können. In "`Up and Down the Ladder of Abstraction"', beschreibt Bret Victor wie sich Systeme in ihrer Ganzheit besser begreifen und gestalten lassen.

		\begin{quote}
			\emph{"`When designing [a system], the challenge lies not in constructing the system, but in understanding it. In the absence of theory, we must develop an intuition to guide our decisions. The design process is thus one of exploration and discovery."'} \parencite{victor}
		\end{quote}

		Auch eine Ansammlung an Daten ist erst einmal sehr abstrakt. In einer Datenbank in Relation gebracht, bleiben die meisten Erkenntnisse und Stories verborgen. Ein tieferes Verständnis, begreifen wir erst dann, wenn wir sie auswerten. Die Art der Datenvisualisierung hat sich in den letzten Jahren stark gewandelt. Während anfangs vor allem Daten in der Form von Häufigkeitsanalysen ausgewertet und als Bar- oder Linechart visualisiert wurden, haben wir heute interaktive Karten um geospartiale Zusammenhänge zu veranschaulichen.

		Diesen Trend von dynamischen Visualisierungen nahmen auch verschiedene namhafte Verkehrsunternehmen auf und wir sahen eine Reihe von Live Karten online gehen. So schuf Beispielsweise die Deutsche Bahn den Zugradar\footnote{\url{http://bahn.de/zugradar}} oder München\footnote{\url{http://s-bahn-muenchen.hafas.de/}} eine Karte für ihr S-Bahn Netz. Während sich diese Visualisierungen weitestgehend nur auf die eigenen verfügbaren Daten stützen, schlossen sie eine gesamte Erfassung des öffentlichen Verkehrs aus. 2014 ging Tracker Geops\footnote{\url{http://tracker.geops.de/}} online um diese Lücke zu schließen und bietet mit über 650 integrierten Fahrplänen eine große Abdeckung. Zusätzlich sei noch LiveMap24 \footnote{\url{https://www.livemap24.com/}} von Verdict erwähnt. Auch eine Live Karte, deren Veröffentlichungsdatum mir allerdings nicht bekannt ist.

		Der Vorteil einer Digitalen Karte besteht in seiner ständigen Aktualität. So kann ein statischer Fahrplan keine Informationen zu Störungen, Verspätungen oder Ausfall eines Zuges geben. Auf einer Live Karte lassen sich solche Informationen visuell aufbereiten und dem Anwender vermitteln.
		Auf einer Live Karte lassen sich solche Informationen verarbeiten, auswerten und schließlich für den Anwender individuell bündeln. Der Betrachter kann sehen wie viele Fahrzeuge gerade aktiv sind und kann zusätzlich zum statischen Fahrplan auch visuell erleben wo sich sein Zug oder Bus befindet. Dadurch werden Verspätungen einen visuellen Kontext gegeben. \emph{"`Der Zug hat 5 Minuten Verspätung"'} ist dadurch nicht mehr nur eine Aussage, sondern sie erhält eine Visualisierung die sie erfahrbar macht.
    Für eine Live Karte kommen aber nicht nur Pendler oder Reisende als Nutzer in Frage sondern auch Verkehrsunternehmen oder auch Städte.
    Es lässt sich simulieren welche Frequenz ein Verkehrsnetz aufweist und zu welcher Zeit es besonders aktiv bzw. inaktiv ist. 
		Dadurch lassen sich Erkenntnisse gewinnen, die durch diese Form der Visualisierung Einblicke ermöglichen, die durch eine gedruckte 2D Karte in dieser Art, nicht möglich gewesen währen.\\

		Diese Arbeit befasst sich umfassend mit der Entwicklung einer Live Karte für den öffentlichen Nahverkehr anhand von GTFS Daten. Der Fokus liegt dabei in der Gestaltung von verschiedenen Visualisierungen, welche die User Experience erhöhen und die Karte nicht nur interaktiv, sondern vor allem auch dem Benutzer Freude bei der Bedienung bereitet.
		
		Das verwenden von einer Postgresql Datenbank und der Abfrage von GTFS-Daten stellt dabei eine große Herausforderung dar.

		% Aufzählen der eigenen Features der Karte. Was für Sachen wurden entworfen etc.

		% Abstract: https://www.researchgate.net/publication/224581089_ICE_-_Visual_analytics_for_transportation_incident_datasets


		% \begin{itemize}
		% 	\item Why this topic has its place
		% 	\item Exploration of visualization methods
		% 	\item Finding way of visualizing abstract schedule data into a more readable way
		% 	\item Static timetables provide no information of the current state of a transit network
		% 	\item What is the current state of the art
		% 	\item Show and discuss different visualization approaches
		% 	\item Scope of the thesis (minimal requirements?)
		% 	\item Explain why my work is different and what makes it unique (twist)
		% 	\item My work wants to epxlore various pathes of visalization posibilities and discuss the pros + cons. We rather want to play with the data instead of providing a production ready product.
		% 	\item overview of whats the content of each section
		% \end{itemize}
	% section einleitung
	
	% section eigene_leistung (end)

\end{newpage}