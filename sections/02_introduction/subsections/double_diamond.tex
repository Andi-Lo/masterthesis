\subsection{The Double Diamond}
\label{sub:the_double_diamond}
  Das Projekt wurde nach dem Design Prozess \texttt{"`The Double Diamond"'}\footnote{\url{http://www.designcouncil.org.uk/news-opinion/design-process-what-double-diamond}} bearbeitet. Der Double Diamond wurde vom British Design Council 2005 entwickelt und soll im folgenden danach beschrieben werden.\parencite{designcouncil}

  \begin{figure}[htbp]
    \begin{center}
      \includegraphics[width=0.9\textwidth]{double_diamond}
      \caption{"`The Double Diamond"' eigene Abbildung nach \parencite{designcouncil}}
      \label{fig:double_diamond}
    \end{center}
  \end{figure}

  Der Double Diamond beschreibt ein iterative Prozess. Wie in allen kreativen Prozessen werden dabei eine Reihe von möglichen Ideen geschaffen ("divergentes Denken"), bevor sie verfeinert und auf die beste Idee reduziert werden ("konvergentes Denken"). Der Double Diamond zeigt jedoch an, dass dies zweimal geschieht - einmal zur Bestätigung der Problemdefinition und einmal zur Erstellung der Lösung. Einer der größten Fehler ist es, den linken Diamanten wegzulassen und am Ende das falsche Problem zu lösen.

  \begin{itemize}[label={}]
    \item \textbf{Discover:} Der erste Teil des Double Diamond steht am Anfang des Projektes. Hier wird versucht, die Welt neu zu sehen, Neues wahrzunehmen und Einsichten in das zu lösende Problem zu sammeln.

    \item \textbf{Define:} Der zweite Teil stellt die Definitionsphase dar. Dabei wird versucht alle in der Entdeckungsphase identifizierten Möglichkeiten zu verstehen. Ziel ist es dabei, ein klares Briefing zu entwickeln, das die grundsätzlichen Herausforderungen umrahmt.

    \item \textbf{Develop:} Der dritte Teil markiert eine Entwicklungsphase, in der Lösungen oder Konzepte erstellt, prototypisiert, getestet und iteriert werden. Dieser Prozess des Ausprobierens hilft, Ideen zu verbessern und zu verfeinern.

    \item \textbf{Deliver:} Der letzte Teil des Double Diamond ist die Lieferphase, in der das daraus resultierende Projekt (z. B. ein Produkt, eine Dienstleistung oder eine Umwelt) abgeschlossen, produziert und in Betrieb genommen wird.
  \end{itemize}

% sub:the_double_diamond (end)