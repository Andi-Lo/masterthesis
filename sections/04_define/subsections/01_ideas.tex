\subsection{Kartenart}
\label{ssub:kartenart}
  Aus der Discover-Phase sind verschiedene Ideen bezüglich der Darstellungsarten entstanden. Zum Beispiel sah eine mögliche Lösung vor, die interaktive Karte mit anderen Visualisierungsformen zu kombinieren. So könnten die einzelnen Trips auch als Balkendiagramm dargestellt werden, welches den Fortschritt der zurückzulegenden Strecke verbildlicht. Dabei war angedacht, dass zwischen diesen verschiedenen Visualisierungsformen hin- und hergeschalten werden kann. Auch der Ansatz, dies mit einer Tube-Map zu verbinden, wurde als Idee notiert und war zeitweise als mögliches Ziel definiert. Da sich für Tube-Maps allerdings keine Kartenanbieter fanden, konnte diese Kartenart nicht umgesetzt werden.
  Letztendlich wird angestrebt, eine Classic-Map zu entwerfen, bei welcher die optische Karte subtil gestaltet ist, sodass die animierten Vehicle im Vordergrund stehen. Darüber hinaus sollen dem Anwender einige Standardfunkionen wie Zooming oder Panning zur Interaktion bereitgestellt werden.
  Es wurde entschieden, bei der Entwicklung der Live-Karte zunächst keine spezifische Nutzergruppe anzuvisieren, sondern einen möglichst universellen Prototypen zu entwerfen, welcher vielseitig genutzt und in unterschiedliche Richtungen weiterentwickelt werden könnte.  

% subsubsection kartenart (end)