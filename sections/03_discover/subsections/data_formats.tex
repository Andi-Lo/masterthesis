\subsection{Mögliche Datengrundlage und Format}
\label{sub:mögliche_datengrundlage_und_format}
  Für eine Visualisierung von Öffentlichen Verkehrsmittel sind mehrere Ansätze denkbar. Nachfolgend sollen verschiedene mögliche Ansätze beschrieben werden.

  \subsubsection{Einsatz von GPS Daten}
  \label{ssub:einsatz_von_gps_daten}
    Eine Möglichkeit ist eine GPS basierte Visualisierung der Echtzeitpositionen der Vehicle. Dies ist eine exakte Abbildung des momentanen Zustandes des Verkehrsnetzes und entspricht am nächsten die wirkliche Abbildung des "`Ist"'-Zustands. Die einzelnen Vehicle würden in einem festen Intervall von n-Sekunden ihre Position senden. Diese lässt sich dann auf einer Karte anzeigen. Je nach Frequenz des Aktualisierungsintervalls, ist das Ergebnis der Position genauer oder ungenauer.

  % subsection einsatz_von_gps_daten (end)

    \subsubsection{GTFS Fahrplandaten}
    \label{ssub:gtfs_fahrplandaten}
      Das GTFS (General Transit Feed Specification) ist eine Datenstandardisierung die von Google im Jahr 2006 entwickelt wurde. Vor dessen Einführung gab es weder eine einheitliche Standardisierung, noch ein "`de facto Standard"' für die Fahrpläne des Öffentlichen Nahverkehrs in den USA. Unter anderem aus diesem Grund erfolge die Adaption an das neue GTFS Format sehr schnell. Vor allem für digitale Produkte, wie zum Beispiel Trip-Planer, die viele verschiedene Fahrpläne von unterschiedlichen Unternehmen in ihren Service integrieren müssen, ist ein standardisiertes Datenformat unbedingt notwendig. Heute sind die meisten öffentlich verfügbaren Fahrpläne im GTFS Format auf Plattformen wie: \url{http://transitfeeds.com} oder \url{https://transit.land/} frei verfügbar.\footnote{Momentan besitzt Transitfeeds 535 Feeds (Stand 18.08.2017)}. GTFS ermöglichte es Transit Organisationen ihre Daten für dritte zu öffnen und ist heute das weit verbreitetste offene Datenformat für den öffentlichen Nahverkehr.\parencite[S. 2]{roush}\\

      Ein GTFS Feed besteht aus mindestens 6 und maximal 13 \texttt{csv-Dateien}, die im \texttt{.txt} Format vorliegen müssen. Die Struktur eines Feeds lässt sich in Worten wie folgt beschreiben:

      \begin{quote}
        \textit{Ein GTFS Feed besteht aus einer oder mehreren Routen. Jede Route (\texttt{routes.txt}) hat einen oder mehrere Trips (\texttt{trips.txt}). Jeder Trip besucht eine Abfolge von Stops (\texttt{stops.txt}) zu einer bestimmten Zeit (\texttt{stop\_times.txt}). Trips und Stop-Zeit beinhalten nur die Tageszeit Informationen. Der Kalender (calendar.txt und \texttt{calendar\_dates.txt}) bestimmt dann, an welchen Tagen ein bestimmter Trip stattfindet.} \cite[S. 8]{zervaas}
      \end{quote}

      Um sich den Inhalt einer GTFS-Datei besser vorstellen zu können ist nachfolgend ist ein Auszug der \texttt{stops.txt} Tabelle abgebildet.

      \begin{lstlisting}[captionpos=b, caption=Auszug der ersten Zeilen von \texttt{stops.txt}, label=lst:gtfs-auszug]
        stop_id,stop_name,stop_lat,stop_lon
        668,Moetzingen Bruehlstrasse,48.53249,8.775416
        2840,Ludwigsburg Mainzer Allee,48.9018,9.21601
        6409,Burgholzhof,48.81742,9.191285
      \end{lstlisting}

      Nachfolgend sind kurz die wichtigsten Tabellen beschrieben.

      \begin{itemize}
        \item \texttt{agency.txt}: Beinhaltet Informationen über die Verkehrsunternehmen, welche das Feed und die Daten bereitstellen.

        \item \texttt{routes.txt}: Eine Route ist eine Gruppierung von Trips. Die verschiedenen Eigenschaften einer Route werden in dieser Tabelle gespeichert.

        \item \texttt{trips.txt}: Ein Trip gehört zu einer Route. Eine Route kann dabei beliebig viele Trips haben. Welche Trips aktiv sind wird durch den Kalender festgelegt.

        \item \texttt{calendar.txt}: Bestimmt, an welchen Tagen ein Trip aktiv ist.

        \item \texttt{stop\_times.txt}: Diese Tabelle beschreibt welche Stationen nacheinander angefahren werden. Für jede Station beinhaltet sie die Ankunfts- und Abfahrtszeiten.

        \item \texttt{stops.txt}: Stellt nähere Informationen für jede Station zur Verfügung wie zum Beispiel den Stationsnamen und deren Position.

        \item \texttt{shapes.txt}: Jeder Trip hat eine dazugehörigen Polyline.

      \end{itemize}

      Ein UML-Diagramm, welches die in Relation stehenden Dateien aufzeigt, existiert unter folgender Adresse: \url{https://developers.google.com/transit/gtfs/reference/} 
    
    % subsubsection gtfs_fahrplandaten (end)
    
    \subsubsection{GTFS-Realtime}
    \label{ssub:gtfs_realtime}
      Eine andere Möglichkeit für die Erfassung von Echtzeitpositionen und Verspätungen bietet GTFS-realtime. GTFS-realtime ist ein von Google entwickelter Standard, der Verkehrsunternehmen das Bereitstellen von Echtzeitinformationen ermöglicht. Dabei gibt es 3 verschiedene Feeds die GTFS-realtime zur Verfügung stellt:\parencite{zervaas_realtime}[S. 6]

      1. Vehicle positions\\
      2. Trip updates\\
      3. Service alerts\\

      GTFS-realtime wäre für diese Arbeit deshalb Interessant, da diese Spezifikation Trip Updates und Vehicle-Position Updates ermöglicht. Beispielsweise kann die Interpolation anhand der Verspätung eines Vehicles angepasst werden. Ein Auszug eines Trip Updates ist in Listing~\ref{lst:gtfs_rt_trip_update} zu sehen.

      \begin{lstlisting}[captionpos=b, caption={Auszug eines GTFS-realtime Trip Updates von MBTA},label={lst:gtfs_rt_trip_update}]
  {
  id: 25732950
  trip_update {
    trip {
      trip_id: 25732950,
      start_date: 20150120,
    }
    stop_time_update {
      arival {
        delay: 240
      }
      stop_id: 135
      ...
    }
    ...
  }
  }
    \end{lstlisting}

    Vehicle-Positionen können über ein ähnliches Format bezogen werden Listing~\ref{lst:gtfs_rt_vehicle_position_update}.

    \begin{lstlisting}[captionpos={b},caption={Auszug eines GTFS-realtime Vehicle-Position Updates von MBTA},label={lst:gtfs_rt_vehicle_position_update}]
  {
  id: "v121",
  vehicle {
    trip {
      trip_id: 2590683,
      start_date: 2017017
    },
    position {
      latitude: 42.267967,
      longitude: -71.093834
    },
    ...
  }
  }
    \end{lstlisting}

  Für eine ausführlichere Beschreibung hilft das Buch: \textit{"`The Definitive Guide to GTFS-realtime - How to consume and produce real-time public transportation data with the GTFS-rt specification."'}\parencite{zervaas_realtime} von Quentin Zervaas.\\

  % subsection gtfs_realtime (end)
% subsection mögliche_datengrundlage_und_format (end)