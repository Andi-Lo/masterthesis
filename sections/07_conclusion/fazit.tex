\begin{newpage}
  \section{Fazit}
  \label{sec:fazit}
    In dieser Arbeit wurde aufgezeigt, wie durch die Verwendung von GTFS-Daten eine Live-Visualisierung des Öffentlichen Nahverkehrs für die Region Stuttgart erstellt werden konnte. Durch Technologien wie Mapbox, Nodjes und PostgreSQL wurde eine visuell ansprechende Anwendung entwickelt, die neben der Darstellung aktiver Fahrzeuge im zeitlichen Verlauf weitere Funktionen wie die einer visuellen Wegfindung bietet. Trotz hoher Datenmengen konnten die erwünschten Performance-Zielsetzungen erreicht werden. Der Mehrwert einer solchen interaktiven Karte ergibt sich vor allem für den verkehrsanalytischen Bereich. Hier kann die Karte zur überblicksartigen Erfassung des Verkehrsnetzes und dessen Analyse beitragen. Verkehrsdaten werden auf diese Weise also sowohl erfahr- als auch explorierbar gemacht.\\

    Für den gewöhnlichen Nutzer wäre der momentane Ansatz mit einer gesamten Abbildung des Öffentlichen Nahverkehrs einer Region allerdings auch mit Problemen behaftet. So ist die Übersichtlichkeit durch die Darstellung aller zu einem Zeitpunkt aktiven Fahrzeuge deutlich eingeschränkt und die Identifikation eines Fahrzeugs einer spezifischen Linie auf der Karte wird erschwert. Auch das Fehlen von GTFS-realtime in Stuttgart würde sich für ein konsumentenorientiertes Produkt nachteilig auswirken, da ohne Echtzeitkomponente viele Informationen, die für eine echte Live-Visualisierung wichtig wären, nicht angeboten werden können. Die Art der Anwendung suggeriert dem Nutzer zwar einen Ist-Zustand, da jedoch tatsächlich nur der Soll-Zustand übermittelt wird, bleiben Fahrplanänderungen oder -abweichungen unberücksichtigt.\\

    Als Ausblick für dieses Projekt wären diverse Verbesserungen möglich. Die Verknüpfung mit einer Echtzeitkomponente würde am meisten Potenzial für Weiterentwicklungen bieten. Die verschiedenen Stadien, die ein Fahrzeug annehmen kann (verspätet, verfrüht, pünktlich), könnten in die UI-Elemente mit einfließen und auch auf der Karte in kreativer Weise verarbeitet werden. Beispielsweise wären Änderungen in der farblichen Darstellung eines Vehicles je nach Pünktlichkeit oder Verspätung möglich. Aber auch statistische Auswertungen wie beispielsweise der Anteil an Verspätungen im Vergleich zum Vortag oder über einen gewissen Zeitraum (zu 34\% um 5 Minuten zu spät) könnten dem Anwender angezeigt werden. Möglich wäre auch, die "`Soll"'-Position, also wo sich das Fahrzeug laut Fahrplan eigentlich befinden sollte, im Vergleich zur tatsächlichen Position, anzuzeigen.

    Darüber hinaus könnte der "`Wegfinder"' aus Kapitel \ref{ssub:wegfindung} als Möglichkeit einer visuellen Wegfindung, Potenzial für Optimierungen und Raum für kreative neue Ideen in sich bergen. Hier könnten Tests mit echten Anwendern zeigen, inwieweit diese Möglichkeit anklang findet oder welche Änderungen für eine verbesserte Nutzbarkeit notwendig wären.

    Schlussendlich wurde aus dem Projekt noch die Erkenntnis gewonnen, dass Live-Karten im Allgemeinen wohl spezifischer auf die Anwendungsbereiche und Bedürfnisse entsprechender Nutzergruppen zugeschnitten sein müssten. Dies lässt sich aus der Zusammenschau der Vor- und Nachteile der in diesem Rahmen entwickelten Webanwendung, aber auch unter Berücksichtigung anderer bestehender Visualisierungen, schlussfolgern. Momentan gibt es zwar bereits verschiedene gute Produkte, die eine Live-Visualisierung des Öffentlichen Nahverkehrs - auch auf breiter Basis - abbilden, aber ich kenne niemanden in meinem Umfeld, der solch eine Anwendung verwendet. Dies kann zum einen daran liegen, dass all diese Anwendungen nicht primär für die native Nutzung per Smartphone - der Hauptnutzerbasis im Bereich Mobilität - entwickelt wurden, sondern eher auf Desktop oder Laptop-Bildschirmen gut darstellbar sind. Zum anderen benötigen unterschiedliche Usergruppen unterschiedliche Informationen. Ein Reisender beispielsweise ist im Zug auf andere Informationen angewiesen als ein Pendler. Ein Besucher in der Stadt findet sich anders zurecht als ein dort ansässiger Einheimischer. An einem öffentlichen Platz sind lokale, stationsbezogene Informationen am meisten relevant. Für Bahnstationen, aber auch für öffentliche Plätze oder gar Cafés, die nahe an einer Haltestelle liegen, könnte sich ein Live-Monitor nach Art von \url{http://transitscreen.com/live/} anbieten, welcher die nächstliegendsten Stationen und Linien anzeigt und dem Anwender umfassende Echtzeitinformationen rund um seinen Standort bietet. Deshalb wäre es bildlich gesprochen erfolgversprechender, anstatt einem großen Schweizer Taschenmesser, lieber viele kleinere Nieschenprodukte zu entwickeln. 

    Die im Rahmen dieser Masterarbeit entwickelte Live-Visualisierung stellt demgegenüber also eher ein Experten-Tool als eine Endanwender-Applikation dar. Allerdings könnte der entstandene Prototyp verwendet werden, um Evaluationen mit verschiedenen Endanwendern für eine nutzerspezifische Weiterentwicklung durchzuführen. Dadurch ließen sich eventuell neue spannende Erkenntnisse für die Entwicklung bisher unbekannter Produkte finden lassen.

  % section fazit (end)
\end{newpage}