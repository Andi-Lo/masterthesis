\begin{newpage}
\section{Anhang}
\label{sec:anhang}

\begin{lstlisting}[captionpos=t, caption=Postgresql Datenbankabfrage von allen aktiven Trips mit ihrem dazugehörigen Linienverlauf, label=lst:get_active_trips_query, language=SQL]
SELECT 
  array_to_json(ARRAY_AGG(array[stop_lat, stop_lon])) AS stops_coords,
  array_to_json(ARRAY_AGG(array[
    CAST ( stops.stop_id AS TEXT ),
    CAST ( stop_times.stop_sequence AS TEXT ),
    stops.stop_name, 
    stop_times.departure_time,
    CAST ( stop_times.departure_time_seconds AS TEXT ),
    stop_times.arrival_time,
    CAST ( stop_times.arrival_time_seconds AS TEXT )
  ] ORDER BY stop_times.stop_sequence)) AS stops_object,
  routes.route_short_name,
  routes.route_long_name,
  routes.route_type,
  routes.route_color,
  routes.route_text_color,
  trips.trip_id, 
  trips.route_id,
  trips.direction_id,
  frequency.start_time, 
  frequency.end_time, 
  frequency.start_time_seconds,
  frequency.end_time_seconds,
  frequency.headway_secs AS frequency_seconds,
  calendar.start_date,
  calendar.end_date,
  agencies.agency_timezone,
  agencies.agency_id,
  agencies.agency_name,
  array_to_json(ARRAY_AGG(array[shape_pt_lat, shape_pt_lon] 
    ORDER BY shape_pt_sequence ASC)) AS shape_coords
FROM gtfs_stop_times AS stop_times

INNER JOIN gtfs_trips AS trips
  ON trips.trip_id = stop_times.trip_id 
  AND trips.agency_key = stop_times.agency_key
    
INNER JOIN gtfs_routes AS routes
  ON routes.route_id = trips.route_id 
  AND routes.agency_key = trips.agency_key
   
INNER JOIN gtfs_agency AS agencies
  ON agencies.agency_id = routes.agency_id 
  AND agencies.agency_key = routes.agency_key

INNER JOIN gtfs_stops AS stops 
  ON stops.stop_id = stop_times.stop_id
  AND stops.agency_key = stop_times.agency_key
  AND NOT EXISTS (
    SELECT 0
    FROM denormalized_max_stop_sequence AS max
    WHERE max.agency_key = stop_times.agency_key
    AND max.trip_id = stop_times.trip_id
    AND max.trip_max = stop_times.stop_sequence
  )

LEFT JOIN gtfs_calendar_dates AS calendar_dates
  ON calendar_dates.service_id = trips.service_id
  AND calendar_dates.agency_key = trips.agency_key
  AND date = '2017-05-10' 
  AND calendar_dates.exception_type = 1

LEFT JOIN gtfs_calendar AS calendar 
  ON trips.service_id = calendar.service_id
  AND calendar.agency_key = trips.agency_key
  AND calendar.thursday = 1
  AND '2017-05-10' BETWEEN calendar.start_date 
  AND calendar.end_date OR calendar_dates.date IS NOT NULL
  AND calendar.start_date <= '2017-05-10'
  AND calendar.end_date >= '2017-05-10'

LEFT JOIN gtfs_frequencies AS frequency
  ON frequency.trip_id = trips.trip_id
  AND frequency.agency_key = trips.agency_key
    
LEFT JOIN gtfs_shapes AS shapes
  ON trips.shape_id = shapes.shape_id
  AND trips.agency_key = shapes.agency_key

WHERE (
  frequency.start_time_seconds >= 25200 
  AND frequency.end_time_seconds <= 90000
  AND shapes.shape_id IS NOT NULL
  AND '2017-05-10' BETWEEN calendar.start_date 
  AND calendar.end_date OR calendar_dates.date IS NOT NULL
) 
AND NOT EXISTS (
  SELECT 0
  FROM gtfs_calendar_dates AS date_exceptions
  WHERE date = '2017-05-10'
  AND date_exceptions.agency_key = trips.agency_key
  AND date_exceptions.service_id = trips.service_id
  AND exception_type = 2
)
GROUP BY 
  routes.route_short_name,
  routes.route_long_name,
  routes.route_type,
  routes.route_color,
  routes.route_text_color,
  trips.trip_id, 
  trips.route_id,
  trips.direction_id,
  frequency.start_time, 
  frequency.end_time, 
  frequency.start_time_seconds,
  frequency.end_time_seconds,
  frequency_seconds,
  calendar.start_date,
  calendar.end_date,
  agencies.agency_timezone,
  agencies.agency_id,
  agencies.agency_name
\end{lstlisting}


\begin{lstlisting}[captionpos=t, caption=Erstellen einer denormalized\_shapes Tabelle, label=lst:sql_aggregate_shape, language=SQL]
-- first we creat the new database table
CREATE TABLE denormalized_shapes (
  agency_key text,
  shape_id INTEGER,
  shape_coords json,
  shape_dist_traveled double precision[]
);

-- generating indexes
CREATE INDEX denormalized_shapes_unique_index 
  ON denormalized_shapes (agency_key, shape_id);

-- insert the shape data into new table
INSERT INTO denormalized_shapes
SELECT
  agency_key,
  shape_id,
  shape_coords,
  shape_dist_traveled
FROM (
  SELECT
    shapes.agency_key,
    shapes.shape_id,
    array_to_json(ARRAY_AGG(array[shape_pt_lon, shape_pt_lat]
      ORDER BY shape_pt_sequence ASC))
      AS shape_coords,
    ARRAY_AGG(shape_dist_traveled
      ORDER BY shape_pt_sequence ASC)
      AS shape_dist_traveled

  FROM gtfs_shapes AS shapes

  GROUP BY
    shapes.agency_key,
    shapes.shape_id
) as shp;
\end{lstlisting}

\begin{lstlisting}[captionpos=t, caption=Erstellen der denormalized\_shapes Tabelle, label=lst:denormalized_shapes, language=SQL]
-- create denormalized_trips table
CREATE TABLE denormalized_trips (
  stops_coords json NOT NULL,
  stops_object json NOT NULL,
  stops_dist_traveled json,
  agency_key TEXT NOT NULL,
  trip_id INTEGER NOT NULL,
  route_id INTEGER NOT NULL,
  service_id INTEGER NOT NULL,
  shape_id INTEGER,
  trip_start_time INTEGER NOT NULL,
  trip_end_time INTEGER NOT NULL,
  direction_id INTEGER,
  route_color TEXT,
  route_long_name TEXT,
  route_short_name TEXT,
  route_desc TEXT,
  route_type INTEGER
);
-- create indexes
CREATE INDEX denormalized_trips_index 
  ON denormalized_trips (agency_key, trip_id, route_id, shape_id);
-- now query the data and insert it into the newly created table
INSERT INTO denormalized_trips
SELECT
  stops_coords,
  stops_object,
  agency_key,
  trip_id,
  route_id,
  service_id,
  shape_id,
  trip_start_time,
  trip_end_time,
  route_color,
  route_long_name,
  route_short_name,
  route_desc,
  route_type
FROM (
  SELECT
    array_to_json(ARRAY_AGG(array[stop_lat, stop_lon]
      ORDER BY stop_times.stop_sequence::int)) AS stops_coords,
    array_to_json(ARRAY_AGG(array[
        stops.stop_id,
        CAST ( stop_times.stop_sequence AS TEXT ),
        stops.stop_name,
        stop_times.departure_time,
        CAST ( stop_times.departure_time_seconds AS TEXT ),
        stop_times.arrival_time,
        CAST ( stop_times.arrival_time_seconds AS TEXT )
    ] ORDER BY stop_times.stop_sequence::int)) AS stops_object,
    trips.agency_key,
    trips.trip_id,
    trips.route_id,
    trips.service_id,
    trips.shape_id,
    min(stop_times.arrival_time_seconds) as trip_start_time,
    max(stop_times.departure_time_seconds) as trip_end_time,
    routes.route_color,
    routes.route_long_name,
    routes.route_short_name,
    routes.route_desc,
    routes.route_type
  FROM gtfs_stop_times AS stop_times

  INNER JOIN gtfs_trips AS trips
    ON trips.trip_id = stop_times.trip_id
    AND trips.agency_key = stop_times.agency_key

  INNER JOIN gtfs_routes AS routes
    ON trips.agency_key = routes.agency_key
    AND routes.route_id = trips.route_id

  INNER JOIN gtfs_stops AS stops
    ON stops.stop_id = stop_times.stop_id
    AND stops.agency_key = stop_times.agency_key

  GROUP BY
    trips.agency_key,
    trips.trip_id,
    trips.route_id,
    trips.service_id,
    trips.shape_id,
    routes.route_color,
    routes.route_long_name,
    routes.route_short_name,
    routes.route_desc,
    routes.route_type
) as trps;
\end{lstlisting}

\begin{lstlisting}[captionpos=t, caption=Abfrage von Trips und dessen Polyline in einer Zeitspanne XY zu einem gegebenen Datum Z, label=lst:query_trips, language=SQL]
SELECT
  trips.stops_coords,
  trips.stops_object,
  trips.stops_dist_traveled,
  shapes.shape_coords,
  trips.route_id,
  trips.shape_id,
  trips.trip_id,
  trips.route_color,
  trips.route_desc,
  trips.route_type,
  trips.route_long_name,
  trips.route_short_name,
  trips.trip_start_time,
  trips.trip_end_time
FROM denormalized_trips AS trips

INNER JOIN gtfs_calendar_dates AS calendar_dates
ON calendar_dates.service_id = trips.service_id
AND calendar_dates.agency_key = trips.agency_key
AND date = 'YYYY-MM-DD'
AND exception_type = 1

-- GTFS STUTTGART-VVS does not make use of the calendar, 
-- if you want to process other feeds aswell you might need
-- to join this table too
--LEFT JOIN gtfs_calendar AS calendar
--  ON trips.service_id = calendar.service_id
--  AND calendar.agency_key = trips.agency_key
--  AND calendar.$date = 1

LEFT JOIN denormalized_shapes AS shapes
  ON trips.shape_id = shapes.shape_id
  AND trips.agency_key = shapes.agency_key

WHERE (
  trip_start_time BETWEEN XY AND XY
  AND shapes.shape_coords IS NOT NULL
)
--AND (
--  'YYYY-MM-DD' BETWEEN calendar.start_date AND calendar.end_date
--  OR calendar_dates.date IS NOT NULL
--)
AND NOT EXISTS (
  SELECT 0
  FROM gtfs_calendar_dates AS date_exceptions
  WHERE date = 'YYYY-MM-DD'
  AND date_exceptions.agency_key = trips.agency_key
  AND date_exceptions.service_id = trips.service_id
  AND exception_type = 2
)
\end{lstlisting}
  

\begin{lstlisting}[captionpos=t, caption=Geojson FeatureCollection als Antwort zu \texttt{/trips/:from,:to}, label=lst:geojson_featurecollection]
{
  2498: {
    "type": "FeatureCollection",
    "features": [24]
  },
  1925: {
    "type": "FeatureCollection",
    "features": [7]
  },
  ...
}
\end{lstlisting}


\begin{lstlisting}[captionpos=t, caption=Shape Feature, label=lst:shape_feature,texcl=true, language=JavaScript, literate={ö}{{\"o}}1]
"features": [
  {
    "type": "Feature",
    "properties": {
      "name": "shape",
      "route_color": "#FF0000",
      "route_type": 700,
      "trip_id": 8784,
      "route_short_name": "701",
      "route_long_name": "Sindelf. Eichholz - ZOB - Böbl. ZOB - Diezenhalde",
      "trip_start": 55560,
      "trip_end": 57780,
      "route_type_color": "#f44336",
      "polyline": "kswv@}x`jH_@f@}@?c@UiJ{AQHiBeAqDcCl@yAjAoB|@iAjBsC^cAOi@yCsA
      _FgAq@MuB@_@YkAa@}A]cDgAUM_FLoa@qBcHq@T{@eCqD_EoDoNiJoJwIaGuHaDqGaCmM?cF|
      @kFvLoSxDaFlG{EpAkA\mAFcBf@q@vA_CO_B[
      m@DiB^eANgAG}APuAbAeARYRm@HyA@}ISyBEiAKsGDsCf]aBmBgP]
      iDDiBNsAtBs@~GiBrAm@tCyEzCqCdDyB~FmC|AWhEY|R]
      rHAxKDnFAdMFvc@lE~JtA|Ex@bD`@vKrBnG~@nDRdBDfONj]z@jKlA`DNvC@zGS|E_@`Dg@zF
      sAxHqAlCa@rDKzIB|FLM\?f@|Fv@dBvC|@\d@s@H_CjCPjA@pKLHqBF_GD_@"
    },
    "geometry": {
      "type": "LineString",
      "coordinates": []
    }
  }
]  
\end{lstlisting}

\begin{lstlisting}[captionpos=t, caption=Station Feature, label=lst:station_feature]
{
  "type": "Feature",
  "properties": {
    "stop_id": "4637",
    "stop_sequence": 1,
    "stop_name": "Sindelf. Eichholz",
    "departure_time_secs": "55560",
    "arrival_time_secs": "55560",
    "secs_between_stops": 60,
    "dist_traveled": 0,
    "delta_dist_traveled": 0
  },
  "geometry": {
    "type": "Point",
    "coordinates": [
      9.002212,
      48.73068
    ]
  }
}
\end{lstlisting}

\begin{lstlisting}[captionpos=t, caption=Station Matching, label=lst:match_station, language=JavaScript]
const _ = require('lodash');
const knn = require('rbush-knn');
const rbush = require('rbush');
const turf = require('turf');

// own implementation of turf.js pointOnLine()
// https://github.com/Turfjs/turf/tree/master/packages/turf-point-on-line
function getPointOnLine(line, pt, units) {
  let coords;

  if (line.type === 'Feature') {
    coords = line.geometry.coordinates;
  } else if (line.type === 'LineString') {
    coords = line.coordinates;
  } else {
    throw new Error('input must be a LineString Feature or Geometry');
  }

  let closestPt = turf.point([Infinity, Infinity], { dist: Infinity });
  const start = turf.point(coords[0]);
  const stop = turf.point(coords[1]);
  start.properties.dist = turf.distance(pt, start, units);
  stop.properties.dist = turf.distance(pt, stop, units);
  // perpendicular
  const heightDistance = Math.max(start.properties.dist, stop.properties.dist);
  const direction = turf.bearing(start, stop);
  const perpendicularPt1 = turf.destination(pt, heightDistance, direction + 90, units);
  const perpendicularPt2 = turf.destination(pt, heightDistance, direction - 90, units);
  const intersect = turf.lineIntersect(
    turf.lineString([perpendicularPt1.geometry.coordinates, perpendicularPt2.geometry.coordinates]),
    turf.lineString([start.geometry.coordinates, stop.geometry.coordinates]));
  let intersectPt = turf.point([Infinity, Infinity], { dist: Infinity });
  if (intersect.features.length > 0) {
    intersectPt = intersect.features[0];
    intersectPt.properties.index = line.properties.index1;
    intersectPt.properties.dist = turf.distance(pt, intersectPt, units);
  }

  if (start.properties.dist < closestPt.properties.dist) {
    closestPt = start;
    closestPt.properties.index = line.properties.index1;
  }
  if (stop.properties.dist < closestPt.properties.dist) {
    closestPt = stop;
    closestPt.properties.index = line.properties.index2;
  }
  if (intersectPt && intersectPt.properties.dist < closestPt.properties.dist) {
    closestPt = intersectPt;
    closestPt.properties.index = line.properties.index1;
  }
  return closestPt;
}

// own implementation of turf.js pointOnLine()
// https://github.com/Turfjs/turf/tree/master/packages/turf-line-slice
function lineSlice(startPt, stopPt, line) {
  const start = turf.point(startPt, { index: 0 });
  let coords;
  if (line.type === 'Feature') {
    coords = line.geometry.coordinates;
  } else if (line.type === 'LineString') {
    coords = line.coordinates;
  } else {
    throw new Error('input must be a LineString Feature or Geometry');
  }
  const ends = [stopPt, start];
  const clipCoords = [ends[0].geometry.coordinates];
  for (let i = ends[0].properties.index + 1; i < ends[1].properties.index + 1; i += 1) {
    clipCoords.push(coords[i]);
  }
  clipCoords.push(ends[1].geometry.coordinates);
  return turf.lineString(clipCoords, line.properties);
}

function fillTree(trips) {
  const tree = rbush(200, ['[0]', '[1]', '[0]', '[1]', '[2]']);
  const coords = [];
  for (let k = 0; k < trips.features[0].geometry.coordinates.length; k += 1) {
    const item = [
      trips.features[0].geometry.coordinates[k][0],
      trips.features[0].geometry.coordinates[k][1],
      { index: k },
    ];
    coords.push(item);
  }
  tree.load(coords);
  return tree;
}

function matchStation(geojson) {
  _.forEach(geojson, (trips) => {
    const shape = _.first(trips.features);
    for (let i = 1; i < trips.features.length; i += 1) {
      const tree = fillTree(trips);
      const lat = trips.features[i].geometry.coordinates[0];
      const lng = trips.features[i].geometry.coordinates[1];
      const neighbours = knn(tree, lat, lng, 2);
      const line = turf.lineString([
        [neighbours[0][0], neighbours[0][1]],
        [neighbours[1][0], neighbours[1][1]]],
        { index1: neighbours[0][2].index, index2: neighbours[1][2].index });
      const pointOnLine = getPointOnLine(line, trips.features[i], 'kilometers');
      const lineDistance = turf.lineDistance(lineSlice(trips.features[0].geometry.coordinates[0], pointOnLine, shape));
      trips.features[i].properties.dist_traveled = lineDistance * 1000;
      trips.features[i].geometry.coordinates = pointOnLine.geometry.coordinates;

      if (i > 1) {
        const deltaDist = trips.features[i].properties.dist_traveled - trips.features[i - 1].properties.dist_traveled;
        trips.features[i].properties.delta_dist_traveled = Math.abs(deltaDist);
      } else {
        trips.features[i].properties.delta_dist_traveled = 0;
      }
    }
  });
  return geojson;
}

\end{lstlisting}

      
\end{newpage}