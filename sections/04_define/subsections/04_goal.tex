\subsection{Zielsetzung}
\label{sub:zielsetzung}
  In dieser Arbeit wird macOS Sierra 10.12.6 auf einem Macbook Pro 2,9 GHz Intel Core i5 mit 16 GB 1867 MHz DDR3 und Intel Iris Graphics 6100 1536 MB eingesetzt. Alle Messungen werden auf dieser Maschine ausgeführt. Das Projekt ist mittels einer \texttt{t2 Medium} AWS-Instanz in der Zone \texttt{EU-West} aufgesetzt. Da das Projekt beim Abschluss dieser Arbeit noch nicht veröffentlicht wurde, konnten nur Messungen auf Localhost durchgeführt werden. Der Zugriff auf den AWS Server / Datenbank konnte nur über VPN erfolgen. Durch das Verbinden auf einen VPN-Server können keine genauen Messungen der Performance des Projektes auf AWS erfolgen, da immer die Netzwerkzeiten bis zum VPN Server miteingeflossen wären. Damit bilden die getätigten Messungen den optimalen Zustand ab, wohingegen ein veröffentlichtes System etwas schlechter abschneiden würde.\\

  Das Hauptziel besteht darin, eine interaktive Karte zu entwickeln, die den Öffentlichen Nahverkehr des Verkehrsverbunds Stuttgart-VVS auf einer Live-Karte visualisiert. Dafür soll ein Prototyp entwickelt werden, der für Demonstrationszwecke eingesetzt werden kann. Weitere visuelle Ziele sollen sich im Prozess durch das Erkunden verschiedener Lösungsansätze ergeben. \\

  Außerdem werden Ziele in Bezug auf die Performance der einzelnen Komponenten (Datenbank, Server, Client) gesetzt. Die Datenbank soll ein GTFS-Feed der Stuttgart-VVS aufnehmen und dessen Daten in $0$ bis $200ms$ bereitstellen können. Des Weiteren soll der Nodejs Server die Daten innerhalb von maximal $100ms$ verarbeitet haben. Das Frontend soll die Vehicle mit 60 FPS rendern können.

% subsection zielsetzung (end)