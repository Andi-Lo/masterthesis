\subsection{Wahl der Datengrundlage}
\label{ssub:wahl_der_datengrundlage}
  Im Abschnitt "`\nameref{ssub:einsatz_von_gps_daten}"' wurde bereits erklärt, wie der Einsatz von GPS-Daten eine Live-Visualisierung ermöglichen könnte. Die einhergehenden Probleme sind allerdings nicht unerheblich und das Fehlen zugänglicher Daten macht eine GPS-basierte Visualisierung zu diesem Zeitpunkt unmöglich.\\

  Aus diesem Grund wird ein GTFS-Feed für die Visualisierung des Stuttgarter Nahverkehrs verwendet und dabei in Kauf genommen, dass eine Echtzeitdarstellung unter Berücksichtigung von Verspätungen, Ausfällen und Umfahrungen nicht möglich ist. Stattdessen wird die Visualisierung auf der Grundlage des Fahrplans und durchschnittlichen Geschwindigkeiten entwickelt.

  Eine genauere Erfassung der Geschwindigkeit wäre zwar wünschenswert, bringt allerdings andere Schwierigkeiten mit sich. Die Erfassung der Geschwindigkeit von jedem Vehicle würde eine hohe Menge an Daten bedeuten, die zwischen Server und Client ausgetauscht werden müssen. Ähnlich wie bei einer GPS-basierten Animation, wäre der Client komplett davon abhängig, ständig Daten zu erhalten. Stelle man sich vor, dass mehrere hundert Anwender eine App benutzen, wäre dies eine enorme Menge an Anfragen \& Antworten. Für Smartphones mit schlechter Verbindung ist dieser Umstand ein großes Problem. Ebenso wie die verwendete Bandbreite und der erhöhte Batterieverbrauch durch das ständige Stellen von Anfragen und der Verarbeitung der Antwort.

  Bei einer Interpolation des Fahrplans mit GTFS-Daten ist hingegen keine ständige Verbindung zum Server nötig. Existiert der relevante Teil des Fahrplans auf dem Gerät des Endnutzers, so kann die Animation anhand dieser Daten erfolgen. Zudem wird das Problem des "`Springens"' umgangen, welches vor allem bei GPS-basierter Animation einen Nachteil darstellt. Durch die Interpolation sind glatte Animationen der Vehicle auf der Karte möglich, sodass die Bewegung von A nach B der realen Fahrbewegung eher entspricht. Dadurch kann der Anwender besser nachvollziehen, was geschieht.
  Eine Lösung für das Problem der fehlenden Echtzeiterfassung ließe sich über den Einsatz von GTFS-realtime erreichen. In dieser Arbeit kann GTFS-realtime allerdings nicht verwendet werden, da zum jetzigen Zeitpunkt\footnote{September, 2017}, der Verkehrsverbund Stuttgart-VVS dies nicht (auch nicht durch ein anderes Format) öffentlich anbietet.
  
% subsubsection wahl_der_datengrundlage (end)